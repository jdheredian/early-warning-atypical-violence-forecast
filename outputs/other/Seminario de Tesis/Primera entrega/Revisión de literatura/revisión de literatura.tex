La literatura identifica factores económicos e institucionales como elementos cruciales detrás de la violencia en Colombia. Estudios recientes indican que los altos niveles de violencia en Colombia no pueden explicarse únicamente con indicadores socioeconómicos tradicionales, como pobreza o desigualdad general. En cambio, factores específicos como el narcotráfico, la minería ilegal y la debilidad institucional han sido determinantes. La relación entre conflicto armado y crimen organizado ha generado condiciones favorables para delitos graves como homicidios, extorsiones, secuestros, masacres y terrorismo.

En particular, \citet{rubio2000} sostienen que los altos niveles de homicidios en Colombia se deben principalmente al narcotráfico y a la debilidad del sistema judicial, lo que reduce significativamente los costos esperados del crimen. \citet{sanchez2003} añaden que la desigualdad en los sectores más pobres de la población urbana está directamente relacionada con mayores tasas de violencia, indicando que la violencia no se distribuye uniformemente, sino que afecta particularmente a los grupos más vulnerables.

Asimismo, \citet{bonilla2009} en una revisión amplia de literatura económica sobre violencia en Colombia, destaca cómo las actividades ilegales vinculadas al narcotráfico generaron rentas elevadas que fortalecieron económicamente a actores armados, quienes utilizaron estos recursos para sostener confrontaciones prolongadas. Además, subraya que los factores institucionales, especialmente la impunidad, son claves en la persistencia y agravamiento de los homicidios.

Igualmente, el secuestro como fenómeno fue analizado por \citet{rubio2003}, quien identifica un aumento desde los años noventa como resultado de incentivos económicos estratégicos para grupos guerrilleros como las FARC y el ELN. Según el autor, estos grupos incrementaron significativamente los secuestros como mecanismo de financiación, especialmente en contextos donde otras fuentes de ingresos eran insuficientes.

Por otro lado, la extorsión ha sido conceptualizada en la literatura como un impuesto criminal que garantiza un flujo constante y seguro de ingresos a grupos armados, particularmente en áreas donde el Estado tiene poca presencia \citet{sanchez2005}. Este fenómeno ha sido facilitado por la informalidad económica, que crea víctimas más vulnerables y menos dispuestas a denunciar por temor o desconfianza hacia la autoridad estatal.

Las masacres han sido identificadas como tácticas estratégicas de grupos armados para ejercer control territorial, eliminar competidores y asegurar recursos económicos clave como tierras productivas o rutas para el narcotráfico \citet{feldmann2009}. Este tipo de violencia genera desplazamiento forzado masivo, lo que a su vez aumenta la consolidación del control territorial por parte de los criminales.

Por su parte, \citet{feldmann2009} mencionan que el terrorismo se ha manifestado mediante atentados estratégicos dirigidos a objetivos civiles e infraestructura fundamental. Este tipo de violencia responde a incentivos económicos y políticos específicos, buscando presionar al Estado, desestabilizar regiones económicamente importantes y proteger intereses financieros ilícitos por parte de los grupos armados.

La identificación de estos factores generales es clave para comprender fenómenos específicos como la violencia atípica. En este sentido, resulta fundamental analizar detalladamente los determinantes de este tipo de violencia para poder entender mejor las dinámicas que busca capturar el SAT.

Así, \citet{deas1995} sostienen que la violencia atípica se presenta cuando los hechos violentos no se pueden explicar solo por la violencia política o la criminalidad común. En el caso de Colombia, se observa una combinación de hechos violentos que surgen en un contexto de debilidad institucional y ausencia del monopolio estatal de la violencia. Esto implica que grupos armados ilegales interactúan con el crimen organizado, generando niveles de violencia superiores.

Por otro lado, \citet{rubio1999} señala que la falta del monopolio estatal y la proliferación de economías ilícitas permiten a los grupos criminales actuar sin control, lo que se traduce en altos índices de violencia. La impunidad y la debilidad institucional facilitan no solo extorsiones y secuestros, sino también actos de terrorismo y masacres en zonas de conflicto. Esta relación entre la debilidad del Estado y el crimen organizado explica por qué ciertos territorios se convierten en focos de violencia atípica \citet{rubio1999, sanchez2001violent}.

Además, la literatura destaca que la violencia atípica es un fenómeno dinámico, influido por choques económicos y cambios en el entorno global. \citet{acemoglu2013} y \citet{angrist2008} demuestran que las variaciones en los precios internacionales de commodities afectan las condiciones de vida en regiones vulnerables, lo que puede aumentar la participación en delitos organizados. Por ejemplo, la caída de los ingresos legales en el sector agrícola se asocia con un incremento de homicidios y extorsiones, mientras que el aumento de los precios del petróleo aumenta la competencia entre grupos armados por el control territorial, generando desde secuestros hasta masacres sistemáticas.

Finalmente, la caracterización de la violencia atípica desde la óptica del crimen organizado permite identificar que sus manifestaciones son diversas y responden a estrategias específicas de actores armados. Para México, \citet{acemoglu2015} analiza cómo las intervenciones estatales contra el narcotráfico pueden debilitar estructuras criminales y crear vacíos que fomentan nuevas redes de extorsión y secuestro. Aunque el estudio haya sido realizado en otro país, evidencia que la violencia agregada no resulta de delitos aislados, especialmente en contextos similares a Colombia, sino de una interacción estratégica en la que los grupos del crimen organizado usan tácticas de terrorismo y masacres para consolidar su poder y controlar mercados ilícitos.

Reconocer la violencia atípica implica que las estrategias de seguridad pública deben ser integrales. Es decir, no basta con acciones policiales o represivas, sino que es necesario fortalecer las instituciones y abordar problemas estructurales como la desigualdad, la exclusión y la corrupción \citet{deas1995}. Debido a esto, entender las causas de violencia atípica es importante porque permite identificar un fenómeno complejo que no se explica únicamente por conflictos ideológicos o delitos aislados, sino por la interacción de factores económicos, sociales y políticos.

En este contexto, \citet{Bazzi2022} analizan si es posible predecir la violencia a nivel local utilizando modelos de aprendizaje automático. Su estudio se centra en Colombia e Indonesia, dos países con largos conflictos y datos detallados sobre violencia y factores de riesgo. A través de modelos avanzados de predicción, buscan anticipar la ocurrencia y escalada de violencia, definida como un aumento de al menos una desviación estándar en comparación con el año anterior.

Para ello, los autores utilizan datos de más de dos décadas, combinando información sobre violencia con variables socioeconómicas, geográficas y políticas. Usan algoritmos  regresión LASSO, bosques aleatorios y gradient boosting para estimar la probabilidad de violencia un año después. Sus modelos logran identificar con precisión zonas con violencia persistente, pero tienen dificultades para predecir incrementos atípicos, lo que sugiere que estos cambios dependen de factores no capturados en los datos utilizados.

El estudio muestra que, aunque los modelos cuantitativos son útiles, no son suficientes para una predicción precisa. Así, una buena estimación de la violencia requiere integrar métodos cuantitativos y cualitativos, permitiendo capturar mejor las complejidades del conflicto. Para mejorar los sistemas de alerta temprana, es clave combinar datos estructurados con información cualitativa sobre actores del conflicto, dinámicas locales y eventos políticos.