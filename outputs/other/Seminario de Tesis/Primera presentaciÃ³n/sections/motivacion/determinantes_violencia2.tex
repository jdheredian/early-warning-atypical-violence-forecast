\begin{frame}{Determinantes de la Violencia en Colombia}
    \begin{itemize}
        \item \textbf{Factores estructurales críticos}:
        \begin{itemize}
            \item \alert{Debilidad institucional}: 
            \begin{itemize}
                \item Bonilla (2009): 75\% de homicidios no resueltos en zonas rurales 
            \end{itemize}
            
            \item \alert{Desigualdad socioeconómica}:
            \begin{itemize}
                \item Bourguignon et al. (2003): +1\% desempleo juvenil $\rightarrow$ +0.8\% homicidios
            \end{itemize}
        \end{itemize}
        
        \vspace{0.4cm}
        
        \item \textbf{Incentivos económicos y choques externos}:
        \begin{itemize}
            \item \alert{Bonanzas de commodities}:
            \begin{itemize}
                \item Dube \& Vargas (2013): +10\% precio petróleo $\rightarrow$ +6.5\% violencia
            \end{itemize}
            
            \item \alert{Narcotráfico}:
            \begin{itemize}
                \item Angrist \& Kugler (2008): Regiones con coca tienen 3x más masacres
            \end{itemize}
        \end{itemize}
    \end{itemize}

\end{frame}