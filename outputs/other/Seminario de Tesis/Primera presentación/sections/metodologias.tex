
\makesection{Metodología}

\begin{frame}{Modelo Teórico: Clasificación Binaria}

    \begin{itemize}
        \item \alert{Aprendizaje Supervisado}: \textbf{Clasificación Binaria}
        $$y = f(X)$$
        \pause
        \item \textbf{Función de probabilidad}:
        \[
        y_t = \begin{cases} 
            1 & \text{si } \text{IACV}_t \geq \bar{X} + \sigma \text{ (atípico)} \\
            0 & \text{si } \text{IACV}_t < \bar{X} + \sigma \text{ (normal)}
        \end{cases}
        \]
        \end{itemize}
\end{frame}





\begin{frame}{Estimación}
    \begin{alertblock}{Objetivo}
        Estimar \( P(y=1|X) \) para activar alertas tempranas con \( \uparrow \) precisión y \( \uparrow \) sensibilidad
    \end{alertblock}
    \pause
    \begin{itemize}
        \item \textbf{Variables explicativas \( X \)}:
        \pause
        \begin{itemize}
            \item \alert{Factores estructurales}:
            \begin{itemize}
                \footnotesize
                \item Desigualdad, presencia estatal, economías ilegales, empleo, presencia grupos armados, educación, ubicación, etc.
            \end{itemize}
            \pause
            \item \alert{Rezagos temporales de violencia}:
            \begin{itemize}
                \footnotesize
                \item IACV, IGC e IA
            \end{itemize}
        
        \end{itemize}
        \end{itemize}
\end{frame}




\begin{frame}{Enfoques de Machine Learning: Elastic Net y Random Forest}
    \begin{columns}[T]
        \begin{column}{0.48\textwidth}
            \begin{alertblock}{1. Elastic Net}
                \begin{itemize}
                    
                    \item \textbf{Qué hace}: 
                    \begin{itemize}
                        
                        \item Regresión lineal con penalización combinada L1 (Lasso) y L2 (Ridge)
                    \end{itemize}
                    \item \textbf{Ventaja}: 
                    \begin{itemize}
                        
                        \item Selección automática de variables y reducción de sobreajuste
                    \end{itemize}
                \end{itemize}
            \end{alertblock}
        \end{column}
        
        \begin{column}{0.48\textwidth}
            \begin{alertblock}{2. Random Forest}
                \begin{itemize}
                    
                    \item \textbf{Qué hace}: 
                    \begin{itemize}
                        
                        \item Ensamble de múltiples árboles de decisión
                    \end{itemize}
                    \item \textbf{Ventaja}: 
                    \begin{itemize}
                        
                        \item Maneja alta dimensionalidad y reduce el sobreajuste
                    \end{itemize}
                \end{itemize}
            \end{alertblock}
        \end{column}
    \end{columns}
\end{frame}

\begin{frame}{Enfoques de Machine Learning: XGBoost y LSTM}
    \begin{columns}[T]
        \begin{column}{0.48\textwidth}
            \begin{alertblock}{3. XGBoost}
                \begin{itemize}
                    
                    \item \textbf{Qué hace}: 
                    \begin{itemize}
                        
                        \item Boosting con n iteraciones + regularización
                    \end{itemize}
                    \item \textbf{Ventaja}: 
                    \begin{itemize}
                        \item Similar a Random Forest
                        \item Eficiencia con datos desbalanceados
                        \item Tratamiento de valores nulos
                    \end{itemize}
                \end{itemize}
            \end{alertblock}
        \end{column}
        
        \begin{column}{0.48\textwidth}
            \begin{alertblock}{4. Redes LSTM}
                \begin{itemize}
                    
                    \item \textbf{Qué hace}: 
                    \begin{itemize}
                        
                        \item Modela dinámicas temporales de forma más efectiva que métodos tradicionales
                    \end{itemize}
                    \item \textbf{Ventaja}: 
                    \begin{itemize}
                        
                        \item Detecta patrones temporales críticos 
                    \end{itemize}
                \end{itemize}
            \end{alertblock}
        \end{column}
    \end{columns}

\end{frame}