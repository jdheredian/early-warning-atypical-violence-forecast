\begin{frame}{¿Cómo se puede complementar el SAT usando Machine Learning?}
    \begin{itemize}
        % --- Parte 1: Por qué ML en violencia atípica ---
        \item \textbf{Violencia atípica: un desafío para métodos tradicionales}
        \begin{itemize}
            \item Factores detonantes: \alert{cambios abruptos} + variables no observables (ej: tensiones políticas locales)
            \item Dificultad para detectar relaciones \alert{no lineales} entre variables
            \item \alert{ML como solución}: Procesa múltiples fuentes + identifica patrones históricos complejos
        \end{itemize}
        \pause
        
        % --- Parte 2: Evidencia desde la literatura ---
        \vspace{0.1cm}
        \item \textbf{Lecciones de estudios recientes: Bazzi et al. (2022)}
        \begin{itemize}
            \item Análisis en Colombia e Indonesia con modelos predictivos:
            \begin{itemize}
                \item \alert{Acierto}: Identificación precisa de hotspots de violencia crónica
                \item \alert{Falla}: Baja precisión en predicción de brotes nuevos
            \end{itemize}
            \item Implicación clave: Combinar \alert{ML} con \alert{análisis cualitativo} del SAT
        \end{itemize}

    \end{itemize}
\end{frame}