\makesection{Definiciones}

\begin{frame}{Violencia agregada: el índice IACV}
    \small
    \begin{block}{¿Qué es el IACV?}
        El \textbf{Índice Agregado de Casos Violentos (IACV)} resume la intensidad de violencia a nivel municipal por trimestre.\\ \alert{Propósito}: Medida de seguridad pública agregada
    \end{block}


        \centering
        \begin{tabular}{lcc}
            \toprule
            \textbf{Delito} & \textbf{Pena (años)} & \textbf{Peso (\%)} \\
            \midrule
            Homicidio & 19.0 & 17.04 \\
            Extorsión & 11.5 & 10.31 \\
            Secuestro & 16.0 & 14.35 \\
            Terrorismo & 15.0 & 13.45 \\
            Masacres & 50.0 & 44.84 \\
            \midrule
            
            \bottomrule
        \end{tabular}


\end{frame}


\begin{frame}{Fórmula del IACV}
    \small
    \begin{alertblock}{Definición}
        \[
        IACV_{t,m} = \frac{0.17 \cdot \text{Hom}_{t,m} + 0.10 \cdot \text{Ext}_{t,m} + 0.14 \cdot \text{Sec}_{t,m} + 0.13 \cdot \text{Terr}_{t,m} + 0.44 \cdot \text{Masc}_{t,m}}{\text{Población}_{t,m}}
        \]
    \end{alertblock}

    \begin{block}{Ponderaciones}
        \begin{itemize}
            \item Reflejan el impacto social relativo de cada tipo de hecho.

        \end{itemize}
    \end{block}
\end{frame}

\begin{frame}{¿Qué otros tipos de violencia definimos?}
    \begin{columns}[T]
        \begin{column}{0.48\textwidth}
            \begin{block}{Índice de Amedrentamiento (IA)}
                \begin{itemize}
                        \item Amenazas
                        \item Tentativas de asesinato y atentados
                        \item Desplazamiento forzado
                        \item Hostigamiento

                \end{itemize}
                \alert{Propósito}: Medir clima de miedo
            \end{block}
        \end{column}
        
        \begin{column}{0.48\textwidth}
            \begin{block}{Índice de Gobernanza Criminal (IGC)}

                    \begin{itemize}
                        \item Confinamientos
                        \item Retenes ilegales
                        \item Paros armados
                        \item Extorsión
                    \end{itemize}
                    \alert{Propósito}: Medir control territorial

            \end{block}
        \end{column}
    \end{columns}
\end{frame}

\begin{frame}{¿Qué es la violencia atípica?}
    \small
    \begin{block}{Inspirado en Bazzi et al. (2022)}
        Se define como un \textbf{nivel de violencia que supera el umbral promedio más una desviación estándar}, con respecto a su historial reciente.
    \end{block}

    \begin{block}{Criterio formal}
        \[
        \text{Violencia Atípica}_{t} = 
        \begin{cases}
            1 & \text{si } IACV_t \geq \bar{IACV}_{t-1:t-12} + \sigma_{t-1:t-12} \\
            0 & \text{en otro caso}
        \end{cases}
        \]
    \end{block}

    \begin{itemize}
        \item Detecta picos inusuales a partir de \textbf{comportamientos históricos del municipio}.
        \item Sirve como variable objetivo para los modelos de predicción.
    \end{itemize}
\end{frame}