\begin{frame}{Metodología}
    \small
    \begin{block}{Objetivo del modelo}
        Predecir la \textbf{probabilidad de que ocurra un evento de violencia atípica} en un municipio en un trimestre determinado.
    \end{block}

    \begin{block}{¿Por qué clasificación y no regresión?}
        \begin{itemize}
            \item El interés está en una variable binaria: violencia atípica (sí/no).
            \item La clasificación permite estimar \textbf{probabilidades asociadas a riesgo}.
            \item Los modelos pueden ser \textbf{ajustados (fine-tuned)} para priorizar sensibilidad o precisión.
            \item Fáciles de integrar a esquemas de priorización o alertas.
        \end{itemize}
    \end{block}

\end{frame}

\begin{frame}{Estimación}
    \begin{alertblock}{Estimar \( P(\text{Violencia Atípica}_{t}=1|\vec{X}) \)}
        \[
    \text{ViolenciaAtípica}_{t,m} = f\left(
        \text{IACV}_{t-1:12,m},\ 
        \text{IGC}_{t-1:4,m},\ 
        \text{IA}_{t-1:4,m},\ 
        X_{t,m},
    \right)
    \]
    \end{alertblock}

    \begin{itemize}
        \item \textbf{Variables explicativas}:

        \begin{itemize}
            \item \alert{Factores estructurales}:
            \begin{itemize}
                \footnotesize
                \item Desigualdad, presencia estatal, economías ilegales, empleo, presencia grupos armados, educación, ubicación, etc.
                \item Si tiene o no cultivos de coca en el periodo pasado
                \item Si tiene o no prescencia de un grupo criminal en el periodo pasado
            \end{itemize}

            \item \alert{Rezagos temporales de violencia}:
            \begin{itemize}
                \footnotesize
                \item IACV, IGC e IA
            \end{itemize}
        
        \end{itemize}
        \end{itemize}
\end{frame}

\begin{frame}{Lasso}
\begin{block}{\textbf{Descripción}}
    \small
            \begin{itemize}
            \item Modelo lineal penalizado que selecciona automáticamente las variables más relevantes.
            \item Supone que \textbf{cada variable influye de forma constante y predecible} sobre el riesgo de violencia atípica: si una variable aumenta, el riesgo cambia proporcionalmente.
            \item Favorece interpretabilidad y evita sobreajuste.
        \end{itemize}
\end{block}
    
    \begin{alertblock}
    \small
    \begin{table}[h!]
    \centering
    \begin{tabular}{lcccccc}
        \toprule
        \% Acierto & \% Sensibilidad & \% Especificidad & AUC &  F1 \\
        \midrule
         72 & 74.3 & 70.7 & 0.811 &  0.660 \\
        \bottomrule
    \end{tabular}
    \end{table}

    \end{alertblock}
\end{frame}




\begin{frame}{Random Forest}
    \small
    \begin{block}{\textbf{Descripción}}
        \begin{itemize}
            \item Conjunto de árboles de decisión entrenados sobre muestras aleatorias.
            \item Captura relaciones \textbf{no lineales} y \textbf{interacciones complejas} entre variables.
            \item No requiere supuestos paramétricos fuertes.

        \end{itemize}
    \end{block}
    \begin{alertblock}
    \small
    \begin{table}[h!]
    \centering
    \begin{tabular}{lcccccc}
        \toprule
        \% Acierto & \% Sensibilidad & \% Especificidad & AUC & F1 \\
        \midrule
         85.4 & 60.4 & 99.7 & 0.898 & 0.750 \\
        \bottomrule
    \end{tabular}
    \end{table}
    \end{alertblock}
\end{frame}