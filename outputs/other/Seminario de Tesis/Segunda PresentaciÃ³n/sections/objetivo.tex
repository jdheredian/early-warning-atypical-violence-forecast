
\makesection{Objetivo de estudio}

\begin{frame}{}
    \vspace{3cm}
    \centering
    \Large ¿Pueden los \alert{modelos de aprendizaje automático} mejorar la detección de \alert{eventos atípicos de violencia} asociados al conflicto armado, y así fortalecer el \alert{Sistema de Alertas Tempranas} de la Defensoría del Pueblo?

\end{frame}

\begin{frame}{Contexto}
    %\medium
    \begin{itemize}
        \item El \textbf{Sistema de Alertas Tempranas (SAT)} es una herramienta institucional de monitoreo y advertencia que identifica situaciones de riesgo frente a violaciones de derechos humanos en el marco del conflicto armado en Colombia.
        
        \vspace{0.4cm}
        \item Estas situaciones de riesgo se pueden entender como \textbf{eventos atípicos de violencia} o \textbf{picos de violencia} en determinados territorios y periodos de tiempo.
        
        \vspace{0.4cm}
        \item La idea de esta tesis es \textbf{cuantificar y pronosticar estos eventos} a través de técnicas de aprendizaje automático, con el fin de \textbf{complementar y fortalecer el SAT}.
    \end{itemize}
\end{frame}