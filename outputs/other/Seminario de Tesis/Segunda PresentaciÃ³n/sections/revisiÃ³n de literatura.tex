\makesection{Revisión de literatura}

\begin{frame}{Revisión de literatura}
    \small
    \begin{block}{¿Qué dice la literatura?}
        \begin{itemize}
            \item \textbf{Econometría tradicional}: relaciones causales entre violencia y pobreza, choques económicos, ausencia estatal.\\
            \textit{Goldstone et al. (2010), Hegre et al. (2013), Dube et al. (2019), Levitt \& Rubio (2000), Bourguignon et al. (2003)}
            
            \item \textbf{Aprendizaje automático}: mejora la precisión, detecta patrones no lineales, predice zonas violentas.\\
            \textit{Muchlinski et al. (2016), Hegre et al. (2019), Bazzi et al. (2022)}

            \item \textbf{Colombia}: énfasis en posconflicto, choques económicos, patrones territoriales.\\
            \textit{Meisel \& Vega (2021), Rojas Guerrero \& Grautoff (2019)}
        \end{itemize}
    \end{block}
\end{frame}

\begin{frame}{Limitaciones y contribución}
    \small
        \begin{block}{¿Qué limitaciones enfrentan?}
        \begin{itemize}
            \item Dificultad para predecir \textbf{eventos atípicos o escaladas inesperadas}.
            \item Falta de \textbf{integración institucional} y problemas de interpretabilidad.
            \item Calidad de datos limita la eficacia en amenazas o extorsión.
        \end{itemize}
    \end{block}
    \begin{alertblock}{¿Qué aporta esta tesis?}
        \begin{itemize}
            \item Se enfoca en detectar \textbf{violencia atípica}, no solo patrones promedio.
            \item Evalúa modelos por su \textbf{capacidad para anticipar escaladas de violencia}.
            \item Integra herramientas cuantitativas al \textbf{SAT de la Defensoría del Pueblo}.
        \end{itemize}
    \end{alertblock}

\end{frame}