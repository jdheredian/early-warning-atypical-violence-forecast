
\makesection{Datos}
\begin{frame}{Fuentes de Datos}
Los datos de violencia provienen de tres fuentes principales:

\begin{itemize}
    \item \textbf{Fiscalía General de la Nación}: Registros mensuales de \alert{homicidio, extorsión, secuestro, terrorismo y masacres} a nivel municipal, mensual (2014-2024).
    \item \textbf{Ministerio de Defensa Nacional}: Serie histórica de los mismos delitos con \alert{mayor cobertura temporal} a nivel municipal, mensual(1997-2024)%, permitiendo análisis de tendencias a largo plazo. Idea para hacer PCA para los datos antes del corte de la fiscalía
    \item \textbf{Jurisdicción Especial para la Paz (JEP)}: Datos sobre \alert{presencia de grupos armados y eventos violentos} (2017-2024), incluyendo desplazamientos, hostigamientos y paros armados.
\end{itemize}
\end{frame}
\begin{frame}{Fuentes de Datos}

Para analizar la relación entre violencia y factores estructurales, se integran:

\begin{itemize}
    \item \textbf{Panel Municipal del CEDE} (2005-2023): Información demográfica, socioeconómica e institucional, como \alert{pobreza, acceso a servicios y programas para víctimas}.
    \item \textbf{Cultivos ilícitos de coca} (1999-2023, Observatorio de Drogas de Colombia): Financiación de grupos armados y la \alert{dinámica del conflicto}.
    \item \textbf{Luminosidad nocturna (VIIRS Nighttime Light)} (2012-2023): Indicador proxy de \alert{actividad económica local}.
\end{itemize}
\end{frame}