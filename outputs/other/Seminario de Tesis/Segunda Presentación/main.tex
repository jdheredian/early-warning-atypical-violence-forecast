
%!TEX TS-program = xelatex
%----------------------------------------------------------------------------------------
%	PACKAGES AND THEMES
%----------------------------------------------------------------------------------------
\documentclass[aspectratio=169,xcolor=dvipsnames, t]{beamer}
\usepackage{fontspec}
\usepackage{unicode-math}
\setmathfont{latinmodern-math.otf}
\usetheme{SimplePlusAIC}
\usepackage{hyperref}
\usepackage{graphicx}
\usepackage{booktabs}
\usepackage{tikz}
\usepackage{makecell}
\usepackage{wrapfig}
\usepackage[spanish]{babel}
\usepackage[numbers]{natbib}
\usepackage{subcaption}

%----------------------------------------------------------------------------------------
%	TITLE PAGE CONFIGURATION
%----------------------------------------------------------------------------------------

\title[Pronóstico de Violencia Extrema]{Fortalecimiento del SAT mediante aprendizaje de máquinas en la detección de violencia atípica}
\subtitle{Economía y Derecho}

\author[Heredia Niño]{Juan Diego Heredia Niño}
\institute[Universidad de los Andes]{Facultad de Economía \newline Universidad de los Andes}

\date{\today} 

%----------------------------------------------------------------------------------------
%	PRESENTATION SLIDES
%----------------------------------------------------------------------------------------

\begin{document}

\maketitlepage


\makesection{Objetivo de estudio}

\begin{frame}{}
    \vspace{3cm}
    \centering
    \Large ¿Pueden los \alert{modelos de aprendizaje automático} mejorar la detección de \alert{eventos atípicos de violencia} asociados al conflicto armado, y así fortalecer el \alert{Sistema de Alertas Tempranas} de la Defensoría del Pueblo?

\end{frame}

\begin{frame}{Contexto}
    %\medium
    \begin{itemize}
        \item El \textbf{Sistema de Alertas Tempranas (SAT)} es una herramienta institucional de monitoreo y advertencia que identifica situaciones de riesgo frente a violaciones de derechos humanos en el marco del conflicto armado en Colombia.
        
        \vspace{0.4cm}
        \item Estas situaciones de riesgo se pueden entender como \textbf{eventos atípicos de violencia} o \textbf{picos de violencia} en determinados territorios y periodos de tiempo.
        
        \vspace{0.4cm}
        \item La idea de esta tesis es \textbf{cuantificar y pronosticar estos eventos} a través de técnicas de aprendizaje automático, con el fin de \textbf{complementar y fortalecer el SAT}.
    \end{itemize}
\end{frame}

\makesection{Motivación}


\begin{frame}{Evolución de la Violencia en Colombia}
    \begin{figure}[ht]
        \centering
        \includegraphics[width=0.9\textwidth, height=0.7\textheight, keepaspectratio]{images/image 7.png}
    \end{figure}
    \centering
    \footnotesize\textit{Fuente: Datos  MinDefensa. Elaboración propia}
\end{frame}

\begin{frame}{Vulnerabilidad municipal a la violencia}
    \begin{columns}[T]
        % Columna izquierda
        \begin{column}{0.48\textwidth}
            \centering
            \includegraphics[width=\textwidth, height=0.65\textheight, keepaspectratio]{images/image 4.png}
        \end{column}
        
        % Columna derecha
        \begin{column}{0.48\textwidth}
            \centering
            \includegraphics[width=\textwidth, height=0.65\textheight, keepaspectratio]{images/output1.png}
        \end{column}
    \end{columns}
    \vspace{0.2cm}
    \centering
    \footnotesize\textit{Fuente: Datos Panel CEDE y MinDefensa. Elaboración propia}
\end{frame}



\begin{frame}{Motivación}
    \small
    \begin{block}{¿Por qué importa esta pregunta?}
        \begin{itemize}
            \item La violencia atípica tiene \textbf{alto impacto social} y \textbf{poca capacidad de respuesta}.
            \item Las herramientas actuales (SAT) son cualitativas y pueden no detectar patrones emergentes.
        \end{itemize}
    \end{block}
    
    \begin{block}{¿Qué propone esta tesis?}
        \begin{itemize}
            \item Usar \textbf{aprendizaje automático} para predecir eventos críticos.
            \item Complementar el SAT con una herramienta cuantitativa y replicable.
        \end{itemize}
    \end{block}

%    \begin{block}{¿Por qué ahora?}
%        \begin{itemize}
%            \item Mayor disponibilidad de datos detallados sobre violencia y contexto municipal.
%            \item Necesidad urgente de mejorar la prevención frente a conflictos armados y crimen organizado.
%        \end{itemize}
%    \end{block}
\end{frame}

%\begin{frame}{Motivación}
%    \small
%    \begin{itemize}
%        \item La violencia atípica genera altos costos sociales, económicos y humanitarios, especialmente cuando se presenta de forma repentina e intensa, dejando poco tiempo de respuesta institucional.
        
%        \vspace{0.35cm}
%        \item El Sistema de Alertas Tempranas (SAT) ha sido clave para prevenir violaciones a los derechos humanos, pero se basa en metodologías cualitativas que pueden no ser tan rápidas, ni tomar en cuenta patrones no observables.
        
%        \vspace{0.35cm}
%        \item Integrar herramientas de aprendizaje automático permite \textbf{identificar con anticipación eventos atípicos de violencia} apoyando la capacidad de reacción del Estado.
        
%        \vspace{0.35cm}
%        \item Esta tesis busca ofrecer una herramienta cuantitativa que complemente el SAT y \textbf{fortalezca la prevención basada en evidencia}.
%    \end{itemize}
%\end{frame}
\makesection{Resultados preliminares}

\begin{frame}{¿Ayuda el aprendizaje automático?}
    \small
    \begin{block}{Hallazgo principal (preliminar)}
        Los modelos de aprendizaje automático \textbf{mejoran la capacidad de detección de violencia atípica} frente a métodos tradicionales:
        \begin{itemize}
            \item \textbf{Mayor precisión global} (hasta 85.4\% con Random Forest).
            \item \textbf{Mejor equilibrio} entre sensibilidad y precisión (\textit{F1} $\approx$ 0.75).
            \item \textbf{AUC elevado} ($\approx$ 0.89), lo que permite \textbf{priorizar municipios según su riesgo estimado}, abriendo la puerta a un sistema de alerta más focalizado y preventivo.
        \end{itemize}
    \end{block}

    \vspace{0.2cm}
    \begin{block}{Advertencia}
        Estos resultados son \textbf{preliminares} y están sujetos a revisión en etapas posteriores.
    \end{block}
\end{frame}
Los estudios que buscan anticipar la violencia ligada al conflicto armado siguen dos caminos. El primero se centra la persistencia de la violencia donde ya existe, mientras el segundo intenta prever eventos atípicos. Este trabajo pertenece a la segunda línea y se relaciona con varias investigaciones recientes.
\\\\
Los primeros esquemas estadísticos, como el Political Instability Task Force, usaron regresiones logísticas con datos país-año para anticipar guerras civiles (Goldstone et al., 2010). Aquellos modelos fueron pioneros, pero enfrentaron el problema de desbalance de muestra: la mayoría de naciones no entra en guerra cada año. Con información subnacional, Ward et al. (2013) y Hegre, Hultman y Nygård (2019) mejoraron un poco la precisión al usar datos mensuales, aunque advirtieron que casi toda la capacidad predictiva venía de la inercia histórico-espacial.
\\\\
En Colombia, Bazzi, Blattman y Dercon (2019) compararon Lasso, bosques aleatorios, gradiente impulsado y redes neuronales con paneles municipales. Los modelos localizaron bien focos crónicos de violencia, pero fallaron al anticipar estallidos en municipios antes pacíficos. Radford (2022), usando una red ConvLSTM, halló que el aprendizaje profundo baja el error medio pero no reduce las “sorpresas” cuando la violencia cambia de patrón. En suma, los modelos basados solo en datos pasados tienden a predecir continuidad.
\\\\
Para manejar este sesgo, trabajos recientes han diseñado estrategias para el fuerte desbalance de clases que caracteriza a los eventos atípicos. El sistema ViEWS, por ejemplo, añadió el conteo de protestas locales a sus modelos de conflicto en África y así mejoró la detección de focos nuevos (Rød, Hegre, & Leis, 2023). Rojas y Grautoff (2023) usaron bosques aleatorios para pronosticar masacres; equilibraron la muestra con over-/undersampling y lograron un AUC de 0,88, aunque con más falsos positivos que la regresión logística. D’Orazio (2023) aplicó AutoML y, además de la predicción, reportó intervalos de probabilidad para ayudar a tomar decisiones bajo incertidumbre.
\\\\
En el ámbito urbano, Mohler et al. (2011) en Los Ángeles y Barreras et al. (2016) en Bogotá modelaron delitos como “procesos auto-excitantes”, donde un hecho violento aumenta la probabilidad de otro cercano. Estos modelos predicen bien zonas calientes habituales, pero no están diseñados para detectar estallidos inusuales.
\\\\
El presente estudio atiende la brecha señalada por Bazzi et al. (2019) y Radford (2022). En lugar de explicar promedios históricos, se centra en eventos atípicos medidos con tres índices de violencia (IACV, IA e IGC) para 1 102 municipios entre 1997 y 2024. A diferencia de Rojas y Grautoff (2023), que se enfocan en el AUC, aquí se comparan Lasso y bosques aleatorios con validación cruzada y varias métricas a la vez (AUC = 0,90; precisión global = 0,85; falsos positivos = 0,01). El resultado muestra que es posible bajar de forma notable las falsas alarmas sin perder sensibilidad. Además, siguiendo la recomendación de D’Orazio (2023), las probabilidades generadas se integran al protocolo operativo del SAT, lo que demuestra viabilidad de uso institucional.
\\\\
El diseño del panel municipal reconoce que la historia de violencia es un predictor clave (Bazzi et al., 2019), adopta el rebalanceo de clases de Rojas y Grautoff (2023) y emplea la predicción probabilística sugerida por D’Orazio (2023). Así se da crédito explícito a las innovaciones previas.
\\\\
En resumen, la literatura muestra que los modelos tradicionales predicen bien la continuidad del conflicto, pero aún fallan frente a los brotes imprevistos. Este trabajo avanza en la frontera al buscar mejorar la predicción de estos eventos atípicos, reducir de forma considerable los falsos positivos y al insertar el modelo directamente en el SAT, evidenciando que el aprendizaje automático, combinado con información cualitativa y métricas adecuadas, puede mejorar la prioridad territorial de las alertas tempranas en contextos de conflicto armado.


\section{Definciones}

Para evaluar el riesgo de violencia asociada al conflicto armado se construyen tres índices que resumen distintos aspectos del fenómeno.  

\subsection*{Índices de violencia}

El {Índice Agregado de Criminalidad y Violencia (IACV)} combina homicidio, extorsión, secuestro, terrorismo y masacres para cada municipio y trimestre.  
Cada delito se pondera con la pena promedio establecida en el Código Penal (Ley 599 de 2000), de modo que la gravedad legal determine el peso relativo.  

        \begin{tabular}{lcc}
            \toprule
            \textbf{Delito} & \textbf{Pena (años)} & \textbf{Peso (\%)} \\
            \midrule
            Homicidio & 19.0 & 17.04 \\
            Extorsión & 11.5 & 10.31 \\
            Secuestro & 16.0 & 14.35 \\
            Terrorismo & 15.0 & 13.45 \\
            Masacres & 50.0 & 44.84 \\
            \midrule
            \caption{Ponderaciones del IACV según la pena media de cada delito.}
            \bottomrule
        \end{tabular}
\\\\
Formalmente, el IACV para el municipio m en el trimestre $t$ se calcula como

\[
IACV_{t,m} \;=\;
\frac{0.17\,\text{Hom}_{t,m} + 0.10\,\text{Ext}_{t,m} + 0.14\,\text{Sec}_{t,m} + 0.13\,\text{Terr}_{t,m} + 0.44\,\text{Masc}_{t,m}}
{\text{Población}_{t,m}}
\]
\\
El {Índice de Amedrentamiento (IA)} mide hechos que generan miedo o intimidación —amenazas, tentativas de asesinato, atentados, desplazamiento forzado y hostigamiento— cuyo impacto principal es psicológico y social.
\\\\
El {Índice de Gobernanza Criminal (IGC)} captura la capacidad de grupos armados ilegales para imponer reglas y restringir la actividad civil mediante confinamientos, retenes ilegales, paros armados y extorsiones.

\subsection*{Definición de violencia atípica}

Siguiendo a Bazzi et al. (2019), se define la {violencia atípica} como un nivel que excede el promedio y una desviación estándar del propio municipio en los doce trimestres previos:

\[
\text{ViolenciaAtípica}_{t,m} =
\begin{cases}
1 & \text{si } IACV_{t,m} \ge \bar{IACV}_{t-1:t-12,m} + \sigma_{t-1:t-12,m},\\
0 & \text{en otro caso}.
\end{cases}
\]
\\\\
Este umbral convierte la serie continua en una variable binaria que marca trimestres con picos inusuales de violencia. Dichos eventos son poco frecuentes, lo que reproduce el desbalance propio de los fenómenos que el SAT necesita anticipar y constituye la variable objetivo de los modelos predictivos.
\makesection{Definiciones}

\begin{frame}{Violencia agregada: el índice IACV}
    \small
    \begin{block}{¿Qué es el IACV?}
        El \textbf{Índice Agregado de Casos Violentos (IACV)} resume la intensidad de violencia a nivel municipal por trimestre.\\ \alert{Propósito}: Medida de seguridad pública agregada
    \end{block}


        \centering
        \begin{tabular}{lcc}
            \toprule
            \textbf{Delito} & \textbf{Pena (años)} & \textbf{Peso (\%)} \\
            \midrule
            Homicidio & 19.0 & 17.04 \\
            Extorsión & 11.5 & 10.31 \\
            Secuestro & 16.0 & 14.35 \\
            Terrorismo & 15.0 & 13.45 \\
            Masacres & 50.0 & 44.84 \\
            \midrule
            
            \bottomrule
        \end{tabular}


\end{frame}


\begin{frame}{Fórmula del IACV}
    \small
    \begin{alertblock}{Definición}
        \[
        IACV_{t,m} = \frac{0.17 \cdot \text{Hom}_{t,m} + 0.10 \cdot \text{Ext}_{t,m} + 0.14 \cdot \text{Sec}_{t,m} + 0.13 \cdot \text{Terr}_{t,m} + 0.44 \cdot \text{Masc}_{t,m}}{\text{Población}_{t,m}}
        \]
    \end{alertblock}

    \begin{block}{Ponderaciones}
        \begin{itemize}
            \item Reflejan el impacto social relativo de cada tipo de hecho.

        \end{itemize}
    \end{block}
\end{frame}

\begin{frame}{¿Qué otros tipos de violencia definimos?}
    \begin{columns}[T]
        \begin{column}{0.48\textwidth}
            \begin{block}{Índice de Amedrentamiento (IA)}
                \begin{itemize}
                        \item Amenazas
                        \item Tentativas de asesinato y atentados
                        \item Desplazamiento forzado
                        \item Hostigamiento

                \end{itemize}
                \alert{Propósito}: Medir clima de miedo
            \end{block}
        \end{column}
        
        \begin{column}{0.48\textwidth}
            \begin{block}{Índice de Gobernanza Criminal (IGC)}

                    \begin{itemize}
                        \item Confinamientos
                        \item Retenes ilegales
                        \item Paros armados
                        \item Extorsión
                    \end{itemize}
                    \alert{Propósito}: Medir control territorial

            \end{block}
        \end{column}
    \end{columns}
\end{frame}

\begin{frame}{¿Qué es la violencia atípica?}
    \small
    \begin{block}{Inspirado en Bazzi et al. (2022)}
        Se define como un \textbf{nivel de violencia que supera el umbral promedio más una desviación estándar}, con respecto a su historial reciente.
    \end{block}

    \begin{block}{Criterio formal}
        \[
        \text{Violencia Atípica}_{t} = 
        \begin{cases}
            1 & \text{si } IACV_t \geq \bar{IACV}_{t-1:t-12} + \sigma_{t-1:t-12} \\
            0 & \text{en otro caso}
        \end{cases}
        \]
    \end{block}

    \begin{itemize}
        \item Detecta picos inusuales a partir de \textbf{comportamientos históricos del municipio}.
        \item Sirve como variable objetivo para los modelos de predicción.
    \end{itemize}
\end{frame}
Los datos de violencia provienen de tres fuentes principales. La Fiscalía General de la Nación proporciona registros de homicidio, extorsión, secuestro, terrorismo y masacres a nivel municipal con frecuencia mensual desde 2014 hasta 2024. El Ministerio de Defensa Nacional también reporta estas mismas categorías de delitos, pero con una serie temporal más extensa, desde 1997 hasta 2024, permitiendo un análisis de tendencias de largo plazo. 

Adicionalmente, se utilizaran datos de la Jurisdicción Especial para la Paz (JEP) desde 2017 hasta 2024 sobre la presencia de grupos armados, así como eventos relacionados con la violencia y el control territorial, incluyendo homicidios, amenazas, emboscadas, tentativas de asesinato, atentados, desplazamientos forzados, hostigamientos, confinamientos, extorsiones, retenes ilegales y paros armados. Utilizando estas variables se puede mejorar la caracterización de la violencia, ya que no solo considera delitos registrados formalmente, sino también eventos asociados a la dinámica del conflicto armado.

Para contextualizar la violencia dentro de las características estructurales de los municipios, se incorporan datos del Panel Municipal del CEDE, con información anual disponible desde 2005 hasta 2023. Este conjunto de datos contiene variables que capturan diferentes dimensiones demográficas, socioeconómicas e institucionales. Entre ellas se incluyen medidas de pobreza multidimensional, acceso a servicios básicos, educación, condiciones de infraestructura, salud, programas de atención a víctimas del conflicto y características territoriales como la distancia a mercados y la altura sobre el nivel del mar.

Además, se incorpora información sobre cultivos ilícitos de coca desde 1999 hasta 2023, proveniente del Observatorio de Drogas de Colombia del Ministerio de Justicia. La presencia de estos cultivos ha sido relevante en la literatura como un determinante de violencia y financiación de grupos armados. 

Finalmente se utiliza información de luminosidad nocturna obtenida del VIIRS Nighttime Light, con registros mensuales a nivel municipal desde 2012 hasta 2023. La luminosidad nocturna es un proxy ampliamente utilizado en estudios económicos para estimar el nivel de ingreso y desarrollo económico en regiones donde los datos son limitados.


\makesection{Metodología}

\begin{frame}{Modelo Teórico: Clasificación Binaria}

    \begin{itemize}
        \item \alert{Aprendizaje Supervisado}: \textbf{Clasificación Binaria}
        $$y = f(X)$$
        \pause
        \item \textbf{Función de probabilidad}:
        \[
        y_t = \begin{cases} 
            1 & \text{si } \text{IACV}_t \geq \bar{X} + \sigma \text{ (atípico)} \\
            0 & \text{si } \text{IACV}_t < \bar{X} + \sigma \text{ (normal)}
        \end{cases}
        \]
        \end{itemize}
\end{frame}





\begin{frame}{Estimación}
    \begin{alertblock}{Objetivo}
        Estimar \( P(y=1|X) \) para activar alertas tempranas con \( \uparrow \) precisión y \( \uparrow \) sensibilidad
    \end{alertblock}
    \pause
    \begin{itemize}
        \item \textbf{Variables explicativas \( X \)}:
        \pause
        \begin{itemize}
            \item \alert{Factores estructurales}:
            \begin{itemize}
                \footnotesize
                \item Desigualdad, presencia estatal, economías ilegales, empleo, presencia grupos armados, educación, ubicación, etc.
            \end{itemize}
            \pause
            \item \alert{Rezagos temporales de violencia}:
            \begin{itemize}
                \footnotesize
                \item IACV, IGC e IA
            \end{itemize}
        
        \end{itemize}
        \end{itemize}
\end{frame}




\begin{frame}{Enfoques de Machine Learning: Elastic Net y Random Forest}
    \begin{columns}[T]
        \begin{column}{0.48\textwidth}
            \begin{alertblock}{1. Elastic Net}
                \begin{itemize}
                    
                    \item \textbf{Qué hace}: 
                    \begin{itemize}
                        
                        \item Regresión lineal con penalización combinada L1 (Lasso) y L2 (Ridge)
                    \end{itemize}
                    \item \textbf{Ventaja}: 
                    \begin{itemize}
                        
                        \item Selección automática de variables y reducción de sobreajuste
                    \end{itemize}
                \end{itemize}
            \end{alertblock}
        \end{column}
        
        \begin{column}{0.48\textwidth}
            \begin{alertblock}{2. Random Forest}
                \begin{itemize}
                    
                    \item \textbf{Qué hace}: 
                    \begin{itemize}
                        
                        \item Ensamble de múltiples árboles de decisión
                    \end{itemize}
                    \item \textbf{Ventaja}: 
                    \begin{itemize}
                        
                        \item Maneja alta dimensionalidad y reduce el sobreajuste
                    \end{itemize}
                \end{itemize}
            \end{alertblock}
        \end{column}
    \end{columns}
\end{frame}

\begin{frame}{Enfoques de Machine Learning: XGBoost y LSTM}
    \begin{columns}[T]
        \begin{column}{0.48\textwidth}
            \begin{alertblock}{3. XGBoost}
                \begin{itemize}
                    
                    \item \textbf{Qué hace}: 
                    \begin{itemize}
                        
                        \item Boosting con n iteraciones + regularización
                    \end{itemize}
                    \item \textbf{Ventaja}: 
                    \begin{itemize}
                        \item Similar a Random Forest
                        \item Eficiencia con datos desbalanceados
                        \item Tratamiento de valores nulos
                    \end{itemize}
                \end{itemize}
            \end{alertblock}
        \end{column}
        
        \begin{column}{0.48\textwidth}
            \begin{alertblock}{4. Redes LSTM}
                \begin{itemize}
                    
                    \item \textbf{Qué hace}: 
                    \begin{itemize}
                        
                        \item Modela dinámicas temporales de forma más efectiva que métodos tradicionales
                    \end{itemize}
                    \item \textbf{Ventaja}: 
                    \begin{itemize}
                        
                        \item Detecta patrones temporales críticos 
                    \end{itemize}
                \end{itemize}
            \end{alertblock}
        \end{column}
    \end{columns}

\end{frame}


\begin{frame}{}
    \vspace{3.5cm}
    \centering
    \LARGE \alert{\textbf{¡Gracias!}} 

\end{frame}

\begin{frame}{Referencias}

\begin{itemize}
\tiny
    \item Banco Interamericano de Desarrollo. (2017). \textit{Los costos del crimen y la violencia}. Banco Interamericano de Desarrollo. \url{https://doi.org/10.18235/0006383}

    \item Bazzi, S., Blair, R. A., Dube, O., Gudgeon, M., \& Peck, R. (2022). \textit{The promise and pitfalls of conflict prediction: Evidence from Colombia and Indonesia}. \textit{The Review of Economics and Statistics, 104} (6), 1246–1262. \url{https://doi.org/10.1162/rest_a_01172}
    
    \item Bourguignon, F., Núñez, J., \& Sánchez, F. (2003). \textit{What part of the income distribution does inequality affect crime? The case of Colombia}. Documento CEDE, Universidad de los Andes.

    \item Dube, O., \& Vargas, J. F. (2013). \textit{Commodity price shocks and civil conflict: Evidence from Colombia}. \textit{Review of Economic Studies, 80}(4), 1384–1421. \url{https://doi.org/10.1093/restud/rdt009}

    \item Feldmann, A., \& Hinojosa, V. (2009). \textit{Terrorism in Colombia: Logic and sources of a multidimensional and ubiquitous phenomenon}. \textit{Terrorism and Political Violence, 21} (1), 42-61. \url{https://doi.org/10.1080/09546550802544694}

    \item Levitt, S., \& Rubio, M. (2000). \textit{Understanding crime in Colombia and what can be done about it}. Documento de trabajo, Universidad de Chicago/Fedesarrollo.

    \item Londoño, J. L., \& Guerrero, R. (1999). \textit{Violencia en América Latina: Epidemiología y costos}. Banco Interamericano de Desarrollo.

    \item Mejía, D., \& Restrepo, P. (2011). \textit{The economics of the war on illegal drug production and trafficking}. Documento CEDE, Universidad de los Andes.

    \item Rubio, M. (2003). \textit{El rapto de la pesca milagrosa: Breve historia del secuestro en Colombia}. Documentos CEDE No. 2003-36, Universidad de los Andes.

    \item Sánchez, F., \& Chacón, M. (2005). \textit{Conflicto, Estado y descentralización: Disputa armada por el control local, 1974–2002}. Documento CIDER/Universidad de los Andes – LSE.

\end{itemize}
\end{frame}

\end{document}
