\makesection{Resultados preliminares}

\begin{frame}{¿Ayuda el aprendizaje automático?}
    \small
    \begin{block}{Hallazgo principal (preliminar)}
        Los modelos de aprendizaje automático \textbf{mejoran la capacidad de detección de violencia atípica} frente a métodos tradicionales:
        \begin{itemize}
            \item \textbf{Mayor precisión global} (hasta 85.4\% con Random Forest).
            \item \textbf{Mejor equilibrio} entre sensibilidad y precisión (\textit{F1} $\approx$ 0.75).
            \item \textbf{AUC elevado} ($\approx$ 0.89), lo que permite \textbf{priorizar municipios según su riesgo estimado}, abriendo la puerta a un sistema de alerta más focalizado y preventivo.
        \end{itemize}
    \end{block}

    \vspace{0.2cm}
    \begin{block}{Advertencia}
        Estos resultados son \textbf{preliminares} y están sujetos a revisión en etapas posteriores.
    \end{block}
\end{frame}