Los datos de violencia provienen de tres fuentes principales. La Fiscalía General de la Nación proporciona registros de homicidio, extorsión, secuestro, terrorismo y masacres a nivel municipal con frecuencia mensual desde 2014 hasta 2024. El Ministerio de Defensa Nacional también reporta estas mismas categorías de delitos, pero con una serie temporal más extensa, desde 1997 hasta 2024, permitiendo un análisis de tendencias de largo plazo. 
\\\\
Adicionalmente, se utilizaran datos de la Jurisdicción Especial para la Paz (JEP) desde 2017 hasta 2024 sobre la presencia de grupos armados, así como eventos relacionados con la violencia y el control territorial, incluyendo homicidios, amenazas, emboscadas, tentativas de asesinato, atentados, desplazamientos forzados, hostigamientos, confinamientos, extorsiones, retenes ilegales y paros armados. Utilizando estas variables se puede mejorar la caracterización de la violencia, ya que no solo considera delitos registrados formalmente, sino también eventos asociados a la dinámica del conflicto armado.
\\\\
Para contextualizar la violencia dentro de las características estructurales de los municipios, se incorporan datos del Panel Municipal del CEDE, con información anual disponible desde 2005 hasta 2023. Este conjunto de datos contiene variables que capturan diferentes dimensiones demográficas, socioeconómicas e institucionales. Entre ellas se incluyen medidas de pobreza multidimensional, acceso a servicios básicos, educación, condiciones de infraestructura, salud, programas de atención a víctimas del conflicto y características territoriales como la distancia a mercados y la altura sobre el nivel del mar.
\\\\
Además, se incorpora información sobre cultivos ilícitos de coca desde 1999 hasta 2023, proveniente del Observatorio de Drogas de Colombia del Ministerio de Justicia. La presencia de estos cultivos ha sido relevante en la literatura como un determinante de violencia y financiación de grupos armados. 
\\\\
Finalmente se utiliza información de luminosidad nocturna obtenida del VIIRS Nighttime Light, con registros mensuales a nivel municipal desde 2012 hasta 2023. La luminosidad nocturna es un proxy ampliamente utilizado en estudios económicos para estimar el nivel de ingreso y desarrollo económico en regiones donde los datos son limitados.