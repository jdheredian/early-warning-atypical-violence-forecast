La violencia en Colombia representa un desafío para el desarrollo económico y social, con costos que pueden superar el 3\% del PIB anual, recursos que podrían destinarse a sectores productivos y políticas sociales (BID, 2017). Además, estudios como el de Londoño and Guerrero (1999) indican que al considerar efectos indirectos, como pérdidas en productividad, deterioro del capital humano y disminución de la inversión extranjera,  los costos totales pueden llegar al 7.1\% del PIB en América Latina y el Caribe.
\\\\
En este contexto, identificar picos de violencia, o eventos de violencia atípica, se ha vuelto una prioridad para las instituciones responsables de proteger los derechos humanos y mantener la estabilidad social en Colombia. A diferencia de los incidentes violentos comunes, los episodios de violencia atípica se presentan de forma repentina, lo que dificulta una respuesta rápida y adecuada por parte de las autoridades y de la sociedad. Debido a su naturaleza inesperado e intenso, este tipo de violencia tiene un fuerte impacto, desde el desplazamiento forzado y la pérdida de vidas hasta la desorganización social, económica y política en las comunidades afectadas.
\\\\
El Sistema de Alertas Tempranas (SAT) de la Defensoría del Pueblo es un instrumento institucional que emite advertencias sobre situaciones de vulnerabilidad, riesgo inminente y eventos violentos de carácter atípico. Este funciona a través del monitoreo constante de factores asociados con la violencia y el conflicto armado en el territorio, utilizando metodologías cualitativas, como el análisis de indicadores sociales, la recopilación de testimonios y la observación de campo. Con este enfoque, el SAT busca identificar dinámicas sociales y políticas que puedan favorecer el surgimiento de hechos violentos, de modo que las autoridades y los actores sociales puedan actuar de forma coordinada para prevenir o mitigar sus consecuencias.
\\\\
La cobertura geográfica del SAT abarca desde áreas urbanas hasta zonas rurales remotas, lo que exige una coordinación constante con administraciones locales, la fuerza pública y organizaciones de la sociedad civil. Garantizar la fiabilidad de la información y su correcta interpretación en contextos diversos es un desafío considerable. Por ello, el SAT se sustenta en la recolección directa de datos, el seguimiento de fuentes locales y la verificación de testimonios, procesos que requieren una intervención humana significativa y experiencia en terreno, lo cuál puede dejar por fuera del estudio variables no observables utilizando estas metodologías.
\\\\
Así, la pregunta de investigación es: \¿Pueden los modelos de aprendizaje automático mejorar la detección de eventos atípicos de violencia y, con ello, fortalecer el SAT? El aporte de este trabajo a la línea de estudio es doblemente metodológico y empírico. Primero, introduce tres indicadores de violencia, IACV, IA e IGC, que permiten agregar la intensidad del conflicto a lo largo del tiempo y entre los municipios del país. Segundo, incorpora dos algoritmos de aprendizaje automático, Lasso y Bosques Aleatorios, capaces de estimar de manera trimestral la probabilidad de ocurrencia de violencia atípica en cada municipio.
\\\\
Los resultados preliminares muestran que el Bosques Aleatorios eleva el área bajo la curva ROC del enfoque actual de 0,74 a 0,90, incrementa la precisión global de 0,72 a 0,85 y reduce la proporción de falsas alarmas de 0,69 a 0,01, mientras que el Lasso alcanza un AUC de 0,81 con sensibilidad de 0,74 y especificidad de 0,71. La mejora proviene de la capacidad del algoritmo para explotar patrones espacio-temporales y relaciones no lineales que la metodología cualitativa no puede procesar en tiempo real.
\\\\
El estudio aporta así un marco empírico para anticipar eventos de violencia atípica a nivel nacional, demuestra que la combinación de aprendizaje automático y análisis cualitativo puede incrementar de forma sustantiva la precisión del SAT y ofrece evidencia concreta para sustentar la adopción de herramientas cuantitativas en las rutinas de monitoreo de las instituciones de derechos humanos en Colombia.