\subsection*{Definición de violencia}

En este trabajo, la violencia se aborda desde una perspectiva de seguridad pública, dando prioridad a aquellos delitos que afectan de forma significativa la estabilidad social y que impactan directamente la vida, la libertad y la integridad de las personas. Para ello, se han construido tres índices que permiten cuantificar y distinguir las dinámicas de la violencia en Colombia.

El \textbf{Índice de Violencia Agregada (IACV)} se define como la suma ponderada de las tasas de homicidio, extorsión, secuestro, terrorismo y masacres por cada 100.000 habitantes. La elección de estos delitos se basa en su relevancia en términos de seguridad pública y su capacidad para desestabilizar la sociedad a través de efectos directos e indirectos. La ponderación se realiza utilizando las penas promedio establecidas en el Código Penal Colombiano (Ley 599 de 2000) para cada crimen, lo que permite reflejar la gravedad relativa de cada delito dentro del índice. La siguiente tabla muestra las penas promedio y la ponderación asignada a cada delito:

\begin{table}[h]
\centering
\begin{tabular}{lcc}
\hline
\textbf{Delito} & \textbf{Pena Promedio (Años)} & \textbf{Porcentaje Relativo (\%)} \\
\hline
Homicidio     & 19.0 & 17.04 \\
Extorsión     & 11.5 & 10.31 \\
Secuestro     & 16.0 & 14.35 \\
Terrorismo    & 15.0 & 13.45 \\
Masacres      & 50.0 & 44.84 \\
\hline
\textbf{Total} & \textbf{111.5} & \textbf{100} \\
\hline
\end{tabular}
\caption{Penas promedio y ponderación relativa de cada delito.}
\label{tab:pena}
\end{table}

Desde una perspectiva metodológica, la combinación de diferentes delitos en un solo índice requiere consideraciones éticas y técnicas. Aunque cada crimen tiene características e impactos sociales particulares, su agregación se justifica por la necesidad de contar con un indicador cuantificable y útil en análisis espaciales y temporales. La ponderación por la pena promedio sirve como una aproximación al daño percibido por la legislación y el sistema judicial, permitiendo diferenciar delitos de mayor y menor gravedad. No obstante, es importante reconocer que este método no captura todas las externalidades asociadas a cada delito ni las variadas formas en que afectan a las comunidades.

El \textbf{Índice de Amedrentamiento (IA)} mide la ocurrencia de hechos de intimidación y coacción que, aunque no derivan directamente en violencia letal, generan un ambiente de miedo e inseguridad. Este índice incluye amenazas, emboscadas, tentativas de asesinato, atentados, desplazamiento forzado y hostigamiento. Al centrarse en la percepción y los efectos psicológicos de la violencia, el IA permite capturar dinámicas de victimización que no se reflejan siempre en indicadores tradicionales de seguridad, pero que inciden significativamente en la gobernabilidad y el bienestar social.

Por otro lado, el \textbf{Índice de Gobernanza Común (IGC)} evalúa la capacidad de los grupos armados ilegales para imponer reglas y controlar territorios, restringiendo la movilidad y la actividad económica de la población. Este índice agrupa casos de confinamientos, retenes ilegales, paros armados y extorsiones, que evidencian la forma en que los actores armados intentan sustituir la autoridad del Estado en determinadas zonas. A diferencia del IACV, que se centra en delitos de alto impacto individual, el IGC se enfoca en el fenómeno de captura territorial, midiendo la extorsión como una práctica de coerción destinada a consolidar el poder en un territorio específico.